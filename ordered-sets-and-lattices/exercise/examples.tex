\documentclass{article}
  \usepackage{amsmath}
  \usepackage{amssymb}

\begin{document}

\begin{enumerate}
  \item For a relation $R$ on $\mathbf{Z}$,
  define $aRb$ if there is postive integer $r$ such that $a = b^r$.
  Prove that R is a partial ordering of $\mathbf{Z}$.

  \item Prove:
  $
  \forall a, b \in S, a\prec b \implies (a\precsim b \land a \neq b)
  $.

  \item Prove that quassi-ordered is not symmetry: $a\prec b \implies b \not\prec a$.

  \item Prove: Suppose $S$ is finite poset with $n$ elements.
  Then there exists a consistent enumeration $f: S\to \{1, 2, \cdots, n\}$.

  \item Give an example of a finite nonlineraly ordered set $X = (A, R)$ which is
  isomorphic to $Y = (A, R^{-1})$, the set $A$ with the inverse order.

  \item Prove the Principle of Transfinite Induction:
  Let $A$ be a subset of a well-ordered set $S$
  with the following two properties:

  i. $a_0 \in A$, ii. $s(a) \subseteq A \implies a \in A$.

  Then $A = S$.

  \item Let $S$ be a well-ordered set. Let $f: S\to S$ be a similarity mapping of $S$ into $S$.
  Prove that: $\forall a \in S \implies a \precsim f(a)$.

  \item Let $A$ be a well-ordered set. Let $s(A)$ denote the collection of
  all initial segments $s(a)$ of elements $a\in A$ ordered by set inclusion.
  Prove: $A$ is isomorphic to $s(A)$ by showing that the map $f: A\to s(A)$,
  defined by $f(a) = s(a)$, is a similarity mapping of $A$ onto $s(A)$.

  \item Write the dual of each statement:

  i. $(a\land b)\lor c = (b\lor c)\land(c\lor a)$;
  ii. $(a\land b)\lor a = a\land(b\lor a)$.

  \item Prove: Let $L$ be a lattice, then:

  i. $a \land b = a \iff a \lor b = b$;

  ii. The relation $a\precsim b$(defined by $a\land b = a$ or $a\lor b = b$)
  is a partial order on $L$.

  \item Prove: Let $L$ be a finite distributive lattice.
  Then every $a\in L$ can be written uniquely(except for order)
  as the join of irredudant join irreducible elements.

  \item Prove: Let $L$ be a complemented lattice with unique complements.
  Then the join irreducible elements of $L$, other than $0$, are its atoms.

  \item Give a example of an infinite lattice $L$ with finite length.
\end{enumerate}

\end{document}
