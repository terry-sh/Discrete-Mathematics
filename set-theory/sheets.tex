\documentclass{article}
%\usepackage{mathrsfs}
\usepackage{amssymb}
\usepackage{amsmath}

% table padding
\renewcommand{\arraystretch}{1.5}
\begin{document}

\title{Set theory}

\section{Concepts}

\begin{itemize}
  \item Set
  \item Element
  \item Subset
  \item Superset
\end{itemize}

\section{Special sets}

\begin{itemize}
  \item Universal set: U
  \item Empty set: $\emptyset$
  \item Natural numbers: $\mathbb{N}$
  \item Integer: $\mathbb{Z}$
  \item Rational numbers: $\mathbb{Q}$
  \item Real numbers: $\mathbb{R}$
  \item Complex numbers: $\mathbb{C}$
\end{itemize}

\section{Set operations}

\subsection{Union and interception}

\subsubsection{Union}
$$
A \cup B = \{ x \mid x \in A \lor x \in B \}
$$

\subsubsection{Interception}
$$
A \cap B = \{ x \mid x \in A \land x \in B \}
$$

Properties:
\begin{itemize}
  \item $A \cap B \subseteq A, A \cap B \subseteq B$
  \item $A \subseteq A \cup B, B \subseteq A \cup B$
\end{itemize}

Theorem 1.3: For any sets A and B, we have:
\begin{gather*}
A \cap B \subseteq A \subseteq A \cup B \\
A \cap B \subseteq B \subseteq A \cup B
\end{gather*}

Theorem 1.4: The followings are equivalent:
$$
A \subseteq B, A \cap B = A, A \cup B = B
$$

\subsection{Complement}

\subsubsection{Absolute complement}

$$
A^c = \{ x \mid x \in \text{U}, x \not\in A \}
$$

\subsubsection{Relative Complement}

The relative complement of A and B, donoted by $A\setminus B$, is:
$$
A\setminus B = \{ x \mid x \in A, x \not\in B \}
$$

\subsubsection{Symmetric difference}

Symmetric difference of sets A and B, denoted by $A \oplus B$, is:
$$
A \oplus B = \{ x \mid (x \in A \land x \not\in B) \lor (x \in B \land x \not\in A) \}
$$

That is:
\begin{gather*}
A \oplus B = (A \cup B)\setminus(A \cap B) \\
A \oplus B = (A \setminus B) \cup (B \setminus A)
\end{gather*}

\subsection{Fundamental products}

Consider $n$ distinct sets $A_1, A_2, \cdots, A_n$,
a fundanmental product is a set of form

$$
A_1^* \cap A_2^* \cap \cdots \cap A_n^* (A_i^* = A_i \lor A_i^* = A_i^c)
$$

\section{Algebra of sets, duality}

Theorem 1.5: Sets satisfy the laws in table:
\begin{table}
\caption{Laws of the algebra of sets}
\label{Table: 1-1}
\begin{tabular}{l | l | l}
  \hline
  Idempotent laws  & $A \cup A = A$ & $A \cap A = A$ \\
  \hline
  Associative laws & $(A \cup B) \cup C = A \cup (B \cup C)$ & $(A \cap B) \cap C = A \cap (B \cap C)$ \\
  \hline
  Commutative laws & $A \cup B = B \cup A$ & $A \cap B = B \cap A$ \\
  \hline
  Distributive laws
  & $A \cup (B \cap C) = (A \cup B) \cap (A \cup C)$
  & $A \cap (B \cup C) = (A \cap B) \cup (A \cap C)$ \\
  \hline
  Identity laws & $A \cup \emptyset = A$      & $A \cap \text{U} = A$ \\
                & $A \cup \text{U} = \text{U}$ & $A \cap \emptyset = \emptyset$ \\
  \hline
  Involution laws & $(A^c)^c = A$ & \\

  \hline
  Complement laws & $A \cup A^c = \text{U}$ & $A \cap A^c = \emptyset$ \\
                  & $\text{U}^c = \emptyset$ & $\emptyset^c = \text{U}$ \\
  \hline
  DeMorgan's laws & $(A\cup B)^c = A^c \cap B^c$ & $(A\cap B)^c = A^c \cap B^c$\\
  \hline
\end{tabular}
\end{table}

\section{Venn diagrams}

\end{document}
