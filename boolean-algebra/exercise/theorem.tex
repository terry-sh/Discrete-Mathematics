\documentclass{article}
  \usepackage{amsmath}
  \usepackage{amssymb}
  \newcommand{\setcomp}[1] {{#1}^{\mathsf{c}}}

\begin{document}

\title{Prove for theorems of axiomatic boolean algebra}

Prove idempotent laws:
 \begin{equation}
 x + x = x
 \end{equation}

 \begin{equation}
 x * x = x
 \end{equation}

Proof $x + x = x$:
\begin{eqnarray*}
x + x & = & (x + x) * 1 \\
      & = & (x + x) * (x + x') \longleftarrow \text{apply distributive laws} \\
      & = & x + (x * x') \\
      & = & x + 0 \\
      & = & x
\end{eqnarray*}

Key point: $x + (x * x') = (x + x) * (x + x') \implies (x + x) * (x + x') = x + (x * x')$

Proof $x * x = x$:
\begin{eqnarray*}
x * x & = & (x * x) + 0 \\
      & = & (x * x) + (x * x') \\
      & = & x * (x + x') \\
      & = & x * 1 \\
      & = & x
\end{eqnarray*}

Absorption Laws: 1. $x + (x * y) = x$ 2. $x * (x + y) = x$

\begin{eqnarray*}
x + (x * y) & = & (x * 1) + (x * y) \\
            & = & x * (1 + y) \\
            & = & x * 1 \\
            & = & x \\
\end{eqnarray*}

\begin{eqnarray*}
x * (x + y) & = & x * x + x * y \\
            & = & x + (x * y) \\
            & = & x \\
\end{eqnarray*}


Associative Laws: 1. x + (y + z) = (x + y) + z 2. (x * y) * z = x * (y * z)

let $A = [x + (y + z)] * [(x + y) + z]$, then:
\begin{eqnarray*}
A & = & [(x + y) + z] * [x + (y + z)] \\
  & = & [(x + y) + z] * x + [(x + y) + z] * (y + z) \\
  & = & [(x + y) * x + x * z] + \{[(x + y) + z] * y + [(x + y) + z] * z\} \\
  & = & [x + x * z] + \{[(x + y) * y + z * y] + z\} \\
  & = & x + \{(y + z * y) + z\} \\
  & = & x + (y + z) \\
\end{eqnarray*}
but also:
\begin{eqnarray*}
A & = & [x + (y + z)] * [(x + y) + z] \\
  & = & [x + (y + z)] * (x + y) + [x + (y + z)] * z \\
  & = & \{[x + (y + z)] * x + [x + (y + z)] * y\} + [x * z + (y + z) * z]\\
  & = & \{x + [x * y + (y + z) * y]\} + (x * z + z)\\
  & = & [x + (x * y + y)] + z\\
  & = & (x + y) + z \\
\end{eqnarray*}
so: $A = x + (y + z) = (x + y) + z$.

Boundedness Laws: 1. x + 1 = 1, x * 0 = 0
Proof $x + 1 = 1$:
\begin{eqnarray*}
x + 1 & = & (x + 1) * 1 \\
      & = & (x + 1) * (x + x') \\
      & = & x * (x + x') + 1 * (x + x') \\
      & = & (x * x + x * x') + (x + x') \\
      & = & (x * x') + (x + x + x') \\
      & = & 0 + 1 \\
      & = & 1 \\
\end{eqnarray*}

\end{document}
